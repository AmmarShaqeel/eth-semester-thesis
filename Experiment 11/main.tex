\documentclass{article}
\usepackage[english]{babel}
\usepackage{fancyhdr}
\usepackage{soul}
\usepackage{chemformula}


\pagestyle{fancy}
\fancyhf{}
\rhead{07.05.2019}
\lhead{Experiment \#11}
\rfoot{Page \thepage}

\begin{document}
\section*{Experiment 11}

\section{Aim}
Test 1, 2, and 3 hour immersion times and 0.016, 0.008, and 0
004 wt. Measuring resistance "properly" - testing with gold tabs and solder paste and making the changes of evaporating the solvent so that the wires will not swell.

\section{Method}
\subsection{Preparation}
\subsubsection{Ag Precursor Solution}
\begin{itemize}
    \item  $3 \times 1.5$g solution of \ch{AgCF3COO} 35\% wt
\end{itemize}

\subsubsection{Ascorbic Acid}
\begin{enumerate}
    \item 0.016 wt: 20ml
    \item 0.008 wt: 20ml
    \item 0.004 wt: 20ml
\end{enumerate}

\subsubsection{Spandex Fibers}
27 Fibers in total:
\begin{itemize}
    \item $18\times 4$cm
    \item $9\times 3$cm
\end{itemize}

\subsection{Procedure}
\begin{itemize}
    \item Immerse 9 fibers in each precursor solution
    \item \textit{Precursor Immersion Time}: 1 hour
    \item Remove and allow to dry. \underline{Dried on tissue}
    \item Separate the fibers into 3 sets of 9
    \item Immerse each set with a different concentration of Ascorbic acid
    \item \textit{Ascorbic Acid Immersion Time}: 1, 1.5, and 2 hours
    \item Remove the fibers from the Ascorbic acid at the set time intervals (1 $\times$ 3cm and 2 $\times$ 4cm)
    \item Dry, then solder gold tabs onto the 3 cm fiber and 1 $\times$ 4 cm fiber in each set
    \item Measure resistance with tabs for the two soldered tabs and directly with multimeter for the 2nd 4cm fiber
\end{itemize}

\section{Results}
All Nonconducting.

\section{Conclusion}
All the fibers were non-conducting because they were stretched excessively during the soldering process.
\end{document}


