\documentclass{article}
\usepackage[english]{babel}
\usepackage{fancyhdr}
\usepackage{soul}
\usepackage{chemformula}


\pagestyle{fancy}
\fancyhf{}
\rhead{27.05.2019}
\lhead{Experiment \#14}
\rfoot{Page \thepage}

\begin{document}
\section*{Experiment 14}

\section{Aim}
Testing the effect of reimmersion cycles on the conductivity of the fibers.

\section{Method}
\subsection{Preparation}
\subsubsection{Ag Precursor Solution}
\begin{itemize}
    \item  3g solution of \ch{AgCF3COO} 35\% wt
\end{itemize}

\subsubsection{Ascorbic Acid}
\begin{enumerate}
    \item 0.004 wt: 20ml
    \item 0.016 wt: 20ml
    \item 0.032 wt: 20ml
\end{enumerate}

\subsubsection{Spandex Fibers}
36 Fibers in total:
\begin{itemize}
    \item $12\times 3$cm
    \item $12\times 4$cm
    \item $12\times 5$cm
\end{itemize}

\subsection{Procedure}
\begin{itemize}
    \item Immerse fibers in precursor solution
    \item \textit{Precursor Immersion Time}: 1 hour
    \item Remove and allow to dry (on tissue).
    \item Immerse each set of fibers in different Ascorbic acid
    \item \textit{Ascorbic Acid Immersion Time}: 1 hour 
    \item Remove 3 fibers, then repeat cycle with other fibers 
    \item Repeat for 4 cycles
    \item Measure resistances of the fibers
\end{itemize}

\pagebreak
\section{Results}
Measured using LCR meter and direct contact.
\begin{table}[h!]
\centering
\begin{tabular}{ |c|c|c|c| } 
    \hline
    No. Cycles &  1cm / $\Omega$ & 2cm / $\Omega$ & 3cm / $\Omega$  \\
    \hline
    1 & X & 780 & 70 \\
    2 & 7.0 & 14.0 & 15.0\\
    3 & 2.4 & 4.2 & 8.0\\
    4 & 1.3 & 2.1 & 3.2\\
    \hline
\end{tabular}
 \caption{Resistance of Fibers for 0.004 wt}
\label{table:1}
\end{table}

\begin{table}[h!]
\centering
\begin{tabular}{ |c|c|c|c| } 
    \hline
    No. Cycles &  1cm / $\Omega$ & 2cm / $\Omega$ & 3cm / $\Omega$  \\
    \hline
    1 & 570 & 484 & 1.6k \\
    2 & 3.2 & 10.2 & 14.8\\
    3 & 1.7 & 4.8 & 5.8\\
    4 & 1.0 & 2.3 & 4.7\\
    \hline
\end{tabular}
 \caption{Resistance of Fibers for 0.016 wt}
\label{table:2}
\end{table}

\begin{table}[h!]
\centering
\begin{tabular}{ |c|c|c|c| } 
    \hline
    No. Cycles &  1cm / $\Omega$ & 2cm / $\Omega$ & 3cm / $\Omega$  \\
    \hline
    1 & 1.16k & 466 & 600\\
    2 & 7.4 & 11.2 & 14.8\\
    3 & 2.2 & 3.3 & 5.9\\
    4 & 1.1 & 2.6 & 3.2\\
    \hline
\end{tabular}
 \caption{Resistance of Fibers for 0.032 wt}
\label{table:3}
\end{table}

\section{Conclusion}
Experiment seems to show a clear relationship between number of cycles and the resistance of the fibers, also experiment shows that the resistance of the fibers ($\Omega/cm$) becomes more consistent as the number of cycles increases.\smallskip

\noindent
Concentration of Ascorbic acid does not seem to play a major role.

\end{document}


