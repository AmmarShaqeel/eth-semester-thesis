\documentclass{article}
\usepackage[english]{babel}
\usepackage{fancyhdr}
\usepackage{soul}
\usepackage{chemformula}


\pagestyle{fancy}
\fancyhf{}
\rhead{20.06.2019}
\lhead{Experiment \#15}
\rfoot{Page \thepage}

\begin{document}
\section*{Experiment 15}

\section{Aim}
Testing the effect of re-immersion cycles on the conductivity of the fibers.

\section{Method}
\subsection{Preparation}
\subsubsection{Ag Precursor Solution}
\begin{itemize}
    \item  3 $\times$ 2g solution of \ch{AgCF3COO} 35\% wt
\end{itemize}

\subsubsection{Ascorbic Acid}
\begin{enumerate}
    \item 0.016 wt: 80ml
\end{enumerate}

\subsubsection{Spandex Fibers}
36 Fibers in total:
\begin{itemize}
    \item $12\times 3$cm
    \item $12\times 4$cm
    \item $12\times 5$cm
\end{itemize}

\subsection{Procedure}
\begin{itemize}
    \item Divide fibers into 3 sets, and immerse in precursor
    \item \textit{Precursor Immersion Time}: 30 mins
    \item Remove and allow to dry (on tissue).
    \item Immerse each set of fibers in AA for set period of time
    \item \textit{Ascorbic Acid Immersion Times}: 5 mins, 15 mins, 30 min
    \item Remove 3 fibers, then repeat cycle with other fibers 
    \item Repeat for 4 cycles
    \item Measure resistances of the fibers
\end{itemize}

\pagebreak
\section{Results}
Measured using multimeter and direct contact.
\begin{table}[h!]
\centering
\begin{tabular}{ |c|c|c|c| } 
    \hline
    No. Cycles &  1cm / $\Omega$ & 2cm / $\Omega$ & 3cm / $\Omega$  \\
    \hline
    1 & 50k & 3M & 60k \\
    2 & 40 & 0.4k & 60\\
    3 & 12 & 17 & 25\\
    4 & 3.3 & 7.52 & 10.1\\
    \hline
\end{tabular}
 \caption{Resistance of Fibers for 5 minute cycles}
\label{table:1}
\end{table}

\begin{table}[h!]
\centering
\begin{tabular}{ |c|c|c|c| } 
    \hline
    No. Cycles &  1cm / $\Omega$ & 2cm / $\Omega$ & 3cm / $\Omega$  \\
    \hline
    1 & X & 3.6k & 70k \\
    2 & 16.2 & 15 & 50.2\\
    3 & 3 & 8.1 & 6.8\\
    4 & 3.1 & 5.1 & 7.1\\
    \hline
\end{tabular}
 \caption{Resistance of Fibers for 15 minute cycles}
\label{table:2}
\end{table}

\begin{table}[h!]
\centering
\begin{tabular}{ |c|c|c|c| } 
    \hline
    No. Cycles &  1cm / $\Omega$ & 2cm / $\Omega$ & 3cm / $\Omega$  \\
    \hline
    1 & 0.17k & 0.37k & 0.60\\
    2 & 8 & 12 & 17\\
    3 & 3.0 & 3.8 & 4.7\\
    4 & 1.2 & 1.9 & 3.4\\
    \hline
\end{tabular}
 \caption{Resistance of Fibers for 30 minute cycles}
\label{table:3}
\end{table}


\pagebreak
Values were retaken at a later date and there was some variation in the readings as seen below:


\begin{table}[h!]
\centering
\begin{tabular}{ |c|c|c|c| } 
    \hline
    No. Cycles &  1cm / $\Omega$ & 2cm / $\Omega$ & 3cm / $\Omega$  \\
    \hline
    1 & 50k & 450k & 60k \\
    2 & 40 & 100 & 64\\
    3 & 6.7 & 17 & 27\\
    4 & 4.3 & 9.2 & 10.3\\
    \hline
\end{tabular}
 \caption{Resistance of Fibers for 5 minute cycles}
\label{table:4}
\end{table}

\begin{table}[h!]
\centering
\begin{tabular}{ |c|c|c|c| } 
    \hline
    No. Cycles &  1cm / $\Omega$ & 2cm / $\Omega$ & 3cm / $\Omega$  \\
    \hline
    1 & 300k & 700k & 1.6M \\
    2 & 5.1 & 6.3 & 14\\
    3 & 2.1 & 5.2 & 6.1\\
    4 & 2.0 & 4.4 & 4.9\\
    \hline
\end{tabular}
 \caption{Resistance of Fibers for 15 minute cycles}
\label{table:5}
\end{table}

\begin{table}[h!]
\centering
\begin{tabular}{ |c|c|c|c| } 
    \hline
    No. Cycles &  1cm / $\Omega$ & 2cm / $\Omega$ & 3cm / $\Omega$  \\
    \hline
    1 & 40 & 666 & 2.7k\\
    2 & 4.5 & 4.0 & 11.6\\
    3 & 1.7 & 2.1 & 2.5\\
    4 & 1.1 & 1.44 & 2.2\\
    \hline
\end{tabular}
\label{table:6}
\caption{Resistance of Fibers for 30 minute cycles}
\end{table}


\section{Conclusion}
The shorter cycles seem to have a lower performance compared to the longer cycles, but eventually this effect saturates (30 min is comparable to 1 hour).
\end{document}


